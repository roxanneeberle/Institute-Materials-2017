\documentclass[11pt,twoside]{article}\makeatletter

\IfFileExists{xcolor.sty}%
  {\RequirePackage{xcolor}}%
  {\RequirePackage{color}}
\usepackage{colortbl}
\usepackage{wrapfig}
\usepackage{ifxetex}
\ifxetex
  \usepackage{fontspec}
  \usepackage{xunicode}
  \catcode`⃥=\active \def⃥{\textbackslash}
  \catcode`❴=\active \def❴{\{}
  \catcode`❵=\active \def❵{\}}
  \def\textJapanese{\fontspec{IPAMincho}}
  \def\textChinese{\fontspec{HAN NOM A}\XeTeXlinebreaklocale "zh"\XeTeXlinebreakskip = 0pt plus 1pt }
  \def\textKorean{\fontspec{Baekmuk Gulim} }
  \setmonofont{DejaVu Sans Mono}
  
\else
  \IfFileExists{utf8x.def}%
   {\usepackage[utf8x]{inputenc}
      \PrerenderUnicode{–}
    }%
   {\usepackage[utf8]{inputenc}}
  \usepackage[english]{babel}
  \usepackage[T1]{fontenc}
  \usepackage{float}
  \usepackage[]{ucs}
  \uc@dclc{8421}{default}{\textbackslash }
  \uc@dclc{10100}{default}{\{}
  \uc@dclc{10101}{default}{\}}
  \uc@dclc{8491}{default}{\AA{}}
  \uc@dclc{8239}{default}{\,}
  \uc@dclc{20154}{default}{ }
  \uc@dclc{10148}{default}{>}
  \def\textschwa{\rotatebox{-90}{e}}
  \def\textJapanese{}
  \def\textChinese{}
  \IfFileExists{tipa.sty}{\usepackage{tipa}}{}
  \usepackage{times}
\fi
\def\exampleFont{\ttfamily\small}
\DeclareTextSymbol{\textpi}{OML}{25}
\usepackage{relsize}
\RequirePackage{array}
\def\@testpach{\@chclass
 \ifnum \@lastchclass=6 \@ne \@chnum \@ne \else
  \ifnum \@lastchclass=7 5 \else
   \ifnum \@lastchclass=8 \tw@ \else
    \ifnum \@lastchclass=9 \thr@@
   \else \z@
   \ifnum \@lastchclass = 10 \else
   \edef\@nextchar{\expandafter\string\@nextchar}%
   \@chnum
   \if \@nextchar c\z@ \else
    \if \@nextchar l\@ne \else
     \if \@nextchar r\tw@ \else
   \z@ \@chclass
   \if\@nextchar |\@ne \else
    \if \@nextchar !6 \else
     \if \@nextchar @7 \else
      \if \@nextchar (8 \else
       \if \@nextchar )9 \else
  10
  \@chnum
  \if \@nextchar m\thr@@\else
   \if \@nextchar p4 \else
    \if \@nextchar b5 \else
   \z@ \@chclass \z@ \@preamerr \z@ \fi \fi \fi \fi
   \fi \fi  \fi  \fi  \fi  \fi  \fi \fi \fi \fi \fi \fi}
\gdef\arraybackslash{\let\\=\@arraycr}
\def\@textsubscript#1{{\m@th\ensuremath{_{\mbox{\fontsize\sf@size\z@#1}}}}}
\def\Panel#1#2#3#4{\multicolumn{#3}{){\columncolor{#2}}#4}{#1}}
\def\abbr{}
\def\corr{}
\def\expan{}
\def\gap{}
\def\orig{}
\def\reg{}
\def\ref{}
\def\sic{}
\def\persName{}\def\name{}
\def\placeName{}
\def\orgName{}
\def\textcal#1{{\fontspec{Lucida Calligraphy}#1}}
\def\textgothic#1{{\fontspec{Lucida Blackletter}#1}}
\def\textlarge#1{{\large #1}}
\def\textoverbar#1{\ensuremath{\overline{#1}}}
\def\textquoted#1{‘#1’}
\def\textsmall#1{{\small #1}}
\def\textsubscript#1{\@textsubscript{\selectfont#1}}
\def\textxi{\ensuremath{\xi}}
\def\titlem{\itshape}
\newenvironment{biblfree}{}{\ifvmode\par\fi }
\newenvironment{bibl}{}{}
\newenvironment{byline}{\vskip6pt\itshape\fontsize{16pt}{18pt}\selectfont}{\par }
\newenvironment{citbibl}{}{\ifvmode\par\fi }
\newenvironment{docAuthor}{\ifvmode\vskip4pt\fontsize{16pt}{18pt}\selectfont\fi\itshape}{\ifvmode\par\fi }
\newenvironment{docDate}{}{\ifvmode\par\fi }
\newenvironment{docImprint}{\vskip 6pt}{\ifvmode\par\fi }
\newenvironment{docTitle}{\vskip6pt\bfseries\fontsize{18pt}{22pt}\selectfont}{\par }
\newenvironment{msHead}{\vskip 6pt}{\par}
\newenvironment{msItem}{\vskip 6pt}{\par}
\newenvironment{rubric}{}{}
\newenvironment{titlePart}{}{\par }

\newcolumntype{L}[1]{){\raggedright\arraybackslash}p{#1}}
\newcolumntype{C}[1]{){\centering\arraybackslash}p{#1}}
\newcolumntype{R}[1]{){\raggedleft\arraybackslash}p{#1}}
\newcolumntype{P}[1]{){\arraybackslash}p{#1}}
\newcolumntype{B}[1]{){\arraybackslash}b{#1}}
\newcolumntype{M}[1]{){\arraybackslash}m{#1}}
\definecolor{label}{gray}{0.75}
\def\unusedattribute#1{\sout{\textcolor{label}{#1}}}
\DeclareRobustCommand*{\xref}{\hyper@normalise\xref@}
\def\xref@#1#2{\hyper@linkurl{#2}{#1}}
\begingroup
\catcode`\_=\active
\gdef_#1{\ensuremath{\sb{\mathrm{#1}}}}
\endgroup
\mathcode`\_=\string"8000
\catcode`\_=12\relax

\usepackage[a4paper,twoside,lmargin=1in,rmargin=1in,tmargin=1in,bmargin=1in,marginparwidth=0.75in]{geometry}
\usepackage{framed}

\definecolor{shadecolor}{gray}{0.95}
\usepackage{longtable}
\usepackage[normalem]{ulem}
\usepackage{fancyvrb}
\usepackage{fancyhdr}
\usepackage{graphicx}
\usepackage{marginnote}


\renewcommand*{\marginfont}{\itshape\footnotesize}

\def\Gin@extensions{.pdf,.png,.jpg,.mps,.tif}

  \pagestyle{fancy}

\usepackage[pdftitle={A briefe description of the portes, creekes, bayes, and hauens, of the Weast India: translated out of the Castlin tongue by I.F. The originall whereof was directed to the mightie Prince Don Charles, King of Castile, \&c.},
 pdfauthor={Enciso, Martin Fernández de, d. 1525.}]{hyperref}
\hyperbaseurl{}

	 \paperwidth210mm
	 \paperheight297mm
              
\def\@pnumwidth{1.55em}
\def\@tocrmarg {2.55em}
\def\@dotsep{4.5}
\setcounter{tocdepth}{3}
\clubpenalty=8000
\emergencystretch 3em
\hbadness=4000
\hyphenpenalty=400
\pretolerance=750
\tolerance=2000
\vbadness=4000
\widowpenalty=10000

\renewcommand\section{\@startsection {section}{1}{\z@}%
     {-1.75ex \@plus -0.5ex \@minus -.2ex}%
     {0.5ex \@plus .2ex}%
     {\reset@font\Large\bfseries\sffamily}}
\renewcommand\subsection{\@startsection{subsection}{2}{\z@}%
     {-1.75ex\@plus -0.5ex \@minus- .2ex}%
     {0.5ex \@plus .2ex}%
     {\reset@font\Large\sffamily}}
\renewcommand\subsubsection{\@startsection{subsubsection}{3}{\z@}%
     {-1.5ex\@plus -0.35ex \@minus -.2ex}%
     {0.5ex \@plus .2ex}%
     {\reset@font\large\sffamily}}
\renewcommand\paragraph{\@startsection{paragraph}{4}{\z@}%
     {-1ex \@plus-0.35ex \@minus -0.2ex}%
     {0.5ex \@plus .2ex}%
     {\reset@font\normalsize\sffamily}}
\renewcommand\subparagraph{\@startsection{subparagraph}{5}{\parindent}%
     {1.5ex \@plus1ex \@minus .2ex}%
     {-1em}%
     {\reset@font\normalsize\bfseries}}


\def\l@section#1#2{\addpenalty{\@secpenalty} \addvspace{1.0em plus 1pt}
 \@tempdima 1.5em \begingroup
 \parindent \z@ \rightskip \@pnumwidth 
 \parfillskip -\@pnumwidth 
 \bfseries \leavevmode #1\hfil \hbox to\@pnumwidth{\hss #2}\par
 \endgroup}
\def\l@subsection{\@dottedtocline{2}{1.5em}{2.3em}}
\def\l@subsubsection{\@dottedtocline{3}{3.8em}{3.2em}}
\def\l@paragraph{\@dottedtocline{4}{7.0em}{4.1em}}
\def\l@subparagraph{\@dottedtocline{5}{10em}{5em}}
\@ifundefined{c@section}{\newcounter{section}}{}
\@ifundefined{c@chapter}{\newcounter{chapter}}{}
\newif\if@mainmatter 
\@mainmattertrue
\def\chaptername{Chapter}
\def\frontmatter{%
  \pagenumbering{roman}
  \def\thechapter{\@roman\c@chapter}
  \def\theHchapter{\roman{chapter}}
  \def\thesection{\@roman\c@section}
  \def\theHsection{\roman{section}}
  \def\@chapapp{}%
}
\def\mainmatter{%
  \cleardoublepage
  \def\thechapter{\@arabic\c@chapter}
  \setcounter{chapter}{0}
  \setcounter{section}{0}
  \pagenumbering{arabic}
  \setcounter{secnumdepth}{6}
  \def\@chapapp{\chaptername}%
  \def\theHchapter{\arabic{chapter}}
  \def\thesection{\@arabic\c@section}
  \def\theHsection{\arabic{section}}
}
\def\backmatter{%
  \cleardoublepage
  \setcounter{chapter}{0}
  \setcounter{section}{0}
  \setcounter{secnumdepth}{2}
  \def\@chapapp{\appendixname}%
  \def\thechapter{\@Alph\c@chapter}
  \def\theHchapter{\Alph{chapter}}
  \appendix
}
\newenvironment{bibitemlist}[1]{%
   \list{\@biblabel{\@arabic\c@enumiv}}%
       {\settowidth\labelwidth{\@biblabel{#1}}%
        \leftmargin\labelwidth
        \advance\leftmargin\labelsep
        \@openbib@code
        \usecounter{enumiv}%
        \let\p@enumiv\@empty
        \renewcommand\theenumiv{\@arabic\c@enumiv}%
	}%
  \sloppy
  \clubpenalty4000
  \@clubpenalty \clubpenalty
  \widowpenalty4000%
  \sfcode`\.\@m}%
  {\def\@noitemerr
    {\@latex@warning{Empty `bibitemlist' environment}}%
    \endlist}

\def\tableofcontents{\section*{\contentsname}\@starttoc{toc}}
\parskip0pt
\parindent1em
\def\Panel#1#2#3#4{\multicolumn{#3}{){\columncolor{#2}}#4}{#1}}
\newenvironment{reflist}{%
  \begin{raggedright}\begin{list}{}
  {%
   \setlength{\topsep}{0pt}%
   \setlength{\rightmargin}{0.25in}%
   \setlength{\itemsep}{0pt}%
   \setlength{\itemindent}{0pt}%
   \setlength{\parskip}{0pt}%
   \setlength{\parsep}{2pt}%
   \def\makelabel##1{\itshape ##1}}%
  }
  {\end{list}\end{raggedright}}
\newenvironment{sansreflist}{%
  \begin{raggedright}\begin{list}{}
  {%
   \setlength{\topsep}{0pt}%
   \setlength{\rightmargin}{0.25in}%
   \setlength{\itemindent}{0pt}%
   \setlength{\parskip}{0pt}%
   \setlength{\itemsep}{0pt}%
   \setlength{\parsep}{2pt}%
   \def\makelabel##1{\upshape\sffamily ##1}}%
  }
  {\end{list}\end{raggedright}}
\newenvironment{specHead}[2]%
 {\vspace{20pt}\hrule\vspace{10pt}%
  \label{#1}\markright{#2}%

  \pdfbookmark[2]{#2}{#1}%
  \hspace{-0.75in}{\bfseries\fontsize{16pt}{18pt}\selectfont#2}%
  }{}
      \def\TheFullDate{1578 (revised: 2007-06)}
\def\TheID{A00689\makeatother }
\def\TheDate{1578}
\title{A briefe description of the portes, creekes, bayes, and hauens, of the Weast India: translated out of the Castlin tongue by I.F. The originall whereof was directed to the mightie Prince Don Charles, King of Castile, \&c.}
\author{Enciso, Martin Fernández de, d. 1525.}\makeatletter 
\makeatletter
\newcommand*{\cleartoleftpage}{%
  \clearpage
    \if@twoside
    \ifodd\c@page
      \hbox{}\newpage
      \if@twocolumn
        \hbox{}\newpage
      \fi
    \fi
  \fi
}
\makeatother
\makeatletter
\thispagestyle{empty}
\markright{\@title}\markboth{\@title}{\@author}
\renewcommand\small{\@setfontsize\small{9pt}{11pt}\abovedisplayskip 8.5\p@ plus3\p@ minus4\p@
\belowdisplayskip \abovedisplayskip
\abovedisplayshortskip \z@ plus2\p@
\belowdisplayshortskip 4\p@ plus2\p@ minus2\p@
\def\@listi{\leftmargin\leftmargini
               \topsep 2\p@ plus1\p@ minus1\p@
               \parsep 2\p@ plus\p@ minus\p@
               \itemsep 1pt}
}
\makeatother
\fvset{frame=single,numberblanklines=false,xleftmargin=5mm,xrightmargin=5mm}
\fancyhf{} 
\setlength{\headheight}{14pt}
\fancyhead[LE]{\bfseries\leftmark} 
\fancyhead[RO]{\bfseries\rightmark} 
\fancyfoot[RO]{}
\fancyfoot[CO]{\thepage}
\fancyfoot[LO]{\TheID}
\fancyfoot[LE]{}
\fancyfoot[CE]{\thepage}
\fancyfoot[RE]{\TheID}
\hypersetup{linkbordercolor=0.75 0.75 0.75,urlbordercolor=0.75 0.75 0.75,bookmarksnumbered=true}
\fancypagestyle{plain}{\fancyhead{}\renewcommand{\headrulewidth}{0pt}}\makeatother 
\begin{document}

\makeatletter
\noindent\parbox[b]{.75\textwidth}{\fontsize{14pt}{16pt}\bfseries\raggedright\sffamily\selectfont \@title}
\vskip20pt
\par\noindent{\fontsize{11pt}{13pt}\sffamily\itshape\raggedright\selectfont\@author\hfill\TheDate}
\vspace{18pt}
\makeatother
\let\tabcellsep& \frontmatter % image:http://eebo.chadwyck.com/downloadtiff?vid=5891&page=1
% image:http://eebo.chadwyck.com/downloadtiff?vid=5891&page=1
\par
A BRIEFE Description of the Portes, Creekes, Bayes, and Hauens, {\itshape of the Weast India:} Translated out of the Castlin {\itshape tongue by I. F.} The Originall whereof was {\itshape di­rected to the mightie Prince} Don Charles, King of {\itshape Castile, \&c.}\begin{figure}[htbp]
\noindent\end{figure}
\par
Imprinted at London, by {\itshape Henry Bynneman. Anno.} 1578.% image:http://eebo.chadwyck.com/downloadtiff?vid=5891&page=2
% image:http://eebo.chadwyck.com/downloadtiff?vid=5891&page=2

\section*{To the right worshipfull Sir Humfrey Gilbert {\itshape Knight.}}\par
THere came to my hands of late (right vvoorshipfull) a notable peece of vvoorke, of the Portes, and of diuers rare things bothe of the Easte and VVeast {\itshape Indians,} vvritten by {\itshape Martin Fer­nandes Denciso,} aboute {\itshape Anno.} 1518. then Dedicated to {\itshape Don Charles} King of {\itshape Castile,} and after called in aboute tvventie yeares past, for that it reuealed secretes that the Spanish natiō vvas loth to haue knovven to the vvorlde. And finding in the same vvorke the Lon­gitudes and Altitudes of many Ilandes, and of the Portes of the tracte of the firme lande of {\itshape America,} I thought good to trāslate out of Spanish into English some parte of the same Calling to minde, that your vvorship vvas the firste man of our nation that gaue light to our people for the finding out of the north­vvest straight, and that novv you meane in proper person, and that at your ovvne charges, to take some noble voyage and discouerie in hande, to leaue be­hind you renovvne to your family, and honour and profite to your countrie: I coulde not but honour you in harte. And to make some shevv of my good vvill, I desired much to present some thing to you, and vvas sory that I had no notable vvorke of mat­ter of Nauigation to Dedicate vnto you, meete for your so greate vvorthinesse. But yet such as this is, I dedicate it vnto you, besechyng you most hum­bly % image:http://eebo.chadwyck.com/downloadtiff?vid=5891&page=3
 to take the same in good parte, and to vvay the good vvill of the giuer, as very greate persons of highe honour haue done, vvhen little trifels haue bene giuen them by others of lovv degree. And Sir, albeit this small gifte (in respect of ministring any knovvledge to you your self) may seeme nothing, in that you doe vnderstande the tongues, vvherein this and many other knovvledges of high value, lie hid from our Seamen, although not from you: yet this may for our meere English Seamen, Pilotes, Marri­ners, \&c. not acquaynted vvith forrayne tongues, bring greate pleasure (if it fortune our Mariners or any other of our Nation, to be driuen by vvinde, tempeste, currents, or by other chaunce to any of the Ilandes, Portes, Hauens, Bayes or Forelandes mencioned in this Pamphlet,) and so it may also in the voyage, be a meane to keepe them the more frō idlenesse, the Nurce of villany, and to giue them also right good occasion by vvay of example, vpon any nevv Discouerie, to take the Altitude and La­titude, to set dovvne the tracte of the Ilandes, the natures of the soyles, and to note the qualitie of the ayre, the seuerall benefites that the Soyles and the Riuers yeelde, vvith all the discomodities and vvantes that the same places haue, and if our Coun­trie men fortune the rather to be avvaked out of their heauy sleepe vvherein they haue long lien, and the rather hereby be occasioned to shunne be­stiall ignoraunce, and vvith other nations rather late than neuer to make thēselues shine vvith the bright­nesse of knovvledge, let them giue Sir {\itshape Humfrey Gilbert} the thankes, for vvhose sake I translated the same. % image:http://eebo.chadwyck.com/downloadtiff?vid=5891&page=3
 And thus committing your vvorship to the greate Neptune, the greate God of the Christians that ru­leth lande and Sea, I leaue you to your voyage, and to the gouernment of that mightie God, vvho ne­uer plāted in any man so hie courage, vvith so much desire to greate attempts, but to some greate end, as heretofore in many hath bene seene, and as the se­quele in your happie successes no doubt shall be founde, as England and the vvhole vvorld shall out of question vvit­nesse. From London the xiiij. of May. 1578.

\begin{quote}

\begin{quote}
Your worships at commaundement {\itshape Iohn Frampton.}\end{quote}
\end{quote}
\mainmatter % image:http://eebo.chadwyck.com/downloadtiff?vid=5891&page=4
% image:http://eebo.chadwyck.com/downloadtiff?vid=5891&page=4

\section[{A brief description of the Portes Creekes, Bayes, and Hauens of the Weast India.}]{A brief description of the Portes Creekes, Bayes, and Hauens of the Weast India.}\par
FRom the Ilande called {\itshape Fierro,} vnto the Ilande of saint Nicholas, are twoo hun­dreth leagues: the Iland of saint Nicholas lieth South Southweast, and from thence to the Ilande called {\itshape Fuego} are .xl. leagues: that Iland lieth directly South, and from thence to the Cape of S. Iustin, whiche standeth in the o­ther side of the Equinoctiall line, are foure hundred leagues. The cape of S. Austin lieth South Southweast, with the 
	\normalmarginpar
      \marginnote{The cape of saynt Austin.} Iland called {\itshape Fuego,} \& standeth in eight degrees on the other side of the Equinoctiall towardes the South: and from the Cape of S. Austin vnto the gulfe \& riuer of S. Francis are 
	\normalmarginpar
      \marginnote{The riuer of saynt Francis.} fortie fiue leagues, the coast lieth southweast: the riuer of saint Francis standeth in ten degrees towardes the south: this is a good porte \& it hath a good riuer. From that riuer vnto the 
	\normalmarginpar
      \marginnote{The Bay.} Bay of all Sayntes are seuentie leagues: the Baye lieth Southwest, \& by south, in .xiij. degrees, and there remayneth in the middest porte Roiall, whiche is a good porte, and hath good riuers. The Bay of al Saints hath within it certain litle Ilandes, and within these are twoo good riuers. In the dire­ction towardes the coast, the lande lieth lowe, and the people are naked, \& eate bread of rootes. It is a baren countrie. From the Bay of all Saintes to the porte called {\itshape Seguro} are .lxxx. leagues, the coast lying south \& by weast. Porte {\itshape Seguro} stan­deth .m. xvij. degrees, this is a porte and a Riuer, and it is a good porte. From porte {\itshape Seguro} vnto the cape which is beyond {\itshape Golfo Formoso,} that is to say, the fayre gulfe, are one hundreth and ten leagues, and the coast lieth south southweast: and be­tweene these twoo are many dry and shallowe places, whiche % image:http://eebo.chadwyck.com/downloadtiff?vid=5891&page=5
 lieth on sea borde .xx. and .xxx. leagues: and passing them, you shall finde neare the land barres, and sholdes, which haue vpō them three or foure fadome water of deapth, and neare to­wardes the ende of the sholdes, standeth the gulfe of Saint {\itshape Thome,} whereunto adioyneth the gulfe of {\itshape Barrosas.} 
	\normalmarginpar
      \marginnote{Saynt Thome.}\par
And passing those shallow places, there lieth cape {\itshape Formoso,} that is to say, the fayre cape, in .xxij. degrees and a halfe: and beyng paste the fayre cape, there appeareth a gulfe betweene two landes, and it seemeth to haue a league in breadth, and three or foure in length: and at the ende therof is the riuer cal­led {\itshape Delgado,} this is a good porte, a good coūtrie, and good peo­ple, although that they be naked. From this gulfe to Cape {\itshape Frio,} that is to say, the colde cape, are .xvj. leagues: cape {\itshape Frio} standeth in .xxiij. degrees and a halfe, this cape hath before it an Iland adioyning, and the fayre cape an other, \& betwene these other little Ilands. From the colde Cape the coast doth turne to the Northweast and by North .xxv. leagues, \& from thence it turneth Northeast other .xx. leagues, and maketh the Colde cape lyke to an Iland: and betweene him and the lande there is a good gulfe, with many small Ilandes in the 
	\normalmarginpar
      \marginnote{Cold Cape.} middest: and frō this gulfe vnto the gulfe of the {\itshape Mangues} are xx. leagues. That of the {\itshape Mangues} hath two Ilands at the end of it, and it is great, \& hath .xx. leagues at the mouth. From this gulf vnto the riuer of S. Francis are .lxxv. leagues. And that of saint Francis lieth Southweast in .xxv. degrees, and before that of saynt Francis lieth the riuer of the {\itshape Cananca,} whiche is a good riuer, and in the middest of this coast lieth the cape of saint Sebastian, which entreth foure leagues into the sea, and towardes the Easte parte of it, standeth the porte of 
	\normalmarginpar
      \marginnote{Saynt Se­bastian.} {\itshape Gregorio:} and to the weast parte lieth the porte of {\itshape Terrerias,} which is a good gulfe, and bath an Iland in the middest. All this coast hath many litle Ilands before it. From the porte of saynt Francis vnto the riuer {\itshape Delas Bueltas,} that is to say, the croked riuer, are .lx. leagues, the coast lieth South, the croked riuer standeth in .xxix. degrees, and neare to the riuer of saynt % image:http://eebo.chadwyck.com/downloadtiff?vid=5891&page=5
 Francis toward the land lieth an Iland, which hath in lēgth xxv. leagues, and towarde the lande he seemeth in circle ob­lique: and rounde aboute the same is the Sea, and hath [...] {\gap 〈◊〉} leagues in breadth: and towardes the [...] {\gap ••}rme lande lieth the gulfe and riuer called {\itshape Reparo,} and the riuer of {\itshape Bayadas.} From the crooked riuer vnto the Cape saint Mary are .lxxx. leagues. The cape saynt Mary standeth in .xxxv. degrees, and a little past this cape, entreth in a riuer more than twentie leagues of breadth, where be people that do eate mans flesh. This coast is al full of sholdes. And before cape saynt Mary are certaine small Ilandes. In all these coastes from cape saynt Mary vnto cape saint Austin groweth muche Brasill and little o­ther thing of any profite in it.\par
And seing that we haue spoken of the coast that lieth from the cape of saynt Austin towardes the pole Antartike: let vs returne to speake of that whiche is towardes the parte of the pole Artike, whiche is called the North pole. I say that from the cape of saynt Austin vnto the riuer {\itshape Marauon} are three hundreth leagues: {\itshape Marauon} lieth weast, in seuen de­grees 
	\normalmarginpar
      \marginnote{Marauon.} and a halfe, it is a greate riuer, and hath more than .xv. leagues in breadth, and ryght leagues within the land. It hath many Ilandes, and in this riuer within the lande .xl. leagues, there is neare to the sayd riuer a Mountaine, where­vpon groweth trees of Incense, the trees be of a good height, \& the bowes be like to Plum trees, \& the Incense doth hang at them, as the yet doth at the tyles of a house in the winter season when it dothe freese. In this riuer were taken foure {\itshape In­dians} in a small boate called in the {\itshape Indian} language a {\itshape Canoa,} that came downe by the riuer, and there was takē from them two stones of Emeralds, the one of them being as great as a mans hand▪ They sayde that so many dayes iourney goyng vpwarde by the riuer, they founde a rocke of that stone. In likewise there was takē from them two loa[...] {\gap •}s made of floure, whiche was like to cakes of Sope, and it seemed that they were kne[...] {\gap ••}ed with the licour of {\itshape Balsamo.} All this coaste % image:http://eebo.chadwyck.com/downloadtiff?vid=5891&page=6
 from the cape of saynt Austin vnto {\itshape Maranon} is a cleare coast and deepe, but neare to the riuer are certaine sholdes towards the East parte. And by the weast part the riuer is deepe, and it hath a good entrie. From this riuer {\itshape Maranon,} vnto the ri­uer whiche is called the Sea of fresh water, art .xxv. leagues: 
	\normalmarginpar
      \marginnote{The Sea of freshe vvater.} this riuer hath .lx. leagues of breadth at the mouth, and carieth suche great aboundance of water, that it entereth more than xx. leagues into the Sea, and mingleth not it selfe with the salte water: this breadth goeth .xxv. leagues within the land, and after it is deuided into partes, the one going towards the southeast, and the other towardes the southwest. That which goeth towards the southweast is very deepe and of much wa­ter, and hath a chanell half a league of breadth, that a Carrake may goe vp through it: and the tydes be so swifte, that the shippes haue neede of good cabels. The ryuer of this porte is very good, and there haue bene some that haue entered fiftie leagues within it, and haue seene no Mountaynes. The {\itshape In­dians} of this countrey haue their lippes made full of small holes in foure partes, and through those holes be put small rings, and likewise at their eares: and if any man aske of thē where they had their golde, they answere, that goyng vp by the riuer so many dayes iourney, they found certaine moun­taynes that had much of it, and from those mountaynes they brought it when they would haue it, but they made no great accōpt of it, for they neither buy nor sell, and amongst them is nothing but chaunge. In this countrey they eate bread of rootes, and {\itshape Maiz,} and they eate certaine rootes whiche they call {\itshape Aies} and {\itshape Batatas,} but the {\itshape Batatas} be better than the other rootes, and beyng rawe they haue a smell of Chestnuts: they are to be eatē rosted. These {\itshape Indians} do make wine of y\textsuperscript{e} fruite of Date trees, which fruite is yellow in colour, \& is as great as a litle Doues egge, and being in season is good to be eaten, and of him proceedeth good wine, and is preserued for a long time. These kinde of people doe make their houses with vp­per rowmes, and they sleepe in them, as also all their habi­tation % image:http://eebo.chadwyck.com/downloadtiff?vid=5891&page=6
 is in the vpper rowmes, and that whiche is alowe, they leaue it open: and also they vse certayne mantels of cotton wooll, and these they tie at the endes with ropes, and the one ende of the rope they make fast to one parte of the house, and the other ende to the other parte of the house: and in these they lie, and be their beddes, and these kinde of beddes be vsed in all the Occidentall {\itshape India,} and there is not in any parte of {\itshape India} any chambers that the people do vse to lodge in aloft from the ground, nor they make any high rowmes, but onely in this parte of {\itshape India,} and in all other places they make their houses without any loftes or chambers, and they couer their houses with the leaues of Date trees, and of grasse. And from this fresh water Sea vnto {\itshape Paria,} the coast lieth weast northweast, and all full of sholdes that the shippes can not come neare to the land. There is from this riuer to {\itshape Paria} twoo hundreth and fiftie leagues. In this freshe water sea, the tydes doe ebbe and flowe as much as they do in Brytayne, and standeth in sixe degrees and halfe. {\itshape Paria} standeth on the other side of the Equinoctiall towarde the North, in seuen degrees: In {\itshape Paria} 
	\normalmarginpar
      \marginnote{Paria.} the sea floweth but little, and from {\itshape Paria} towardes the weast the sea dothe not flowe. From the entery of the gulfe of {\itshape Pa­ria} vnto the cape that litth towardes the weast, are thirtie fiue leagues, \& from thence the coast turneth towardes the north­east, other .xxxv. leagues, and from thence the coast turneth towardes the weast. Before this gulfe standeth the Ilande 
	\normalmarginpar
      \marginnote{The Tri­nitie.} of the Trinitie, and towardes the weast dothe appeare the gulfe of {\itshape Paria} like to half a round Circle, after the fashion of a Diametre: and at the ende of this circle is the entery in of {\itshape Paria,} and at this entrie there is betweene the land and the I­land right leagues, \& on the other side there is but little space betweene the Iland and the lande, but it is deepe, and hath a good entrie: this Ilande of the Trinitie hath in length .xxv. leagues, and as many in breadth, \& standeth in. viij▪ degrees, and is inhabited of many people, and as yet not vnder sub­iection. Here the {\itshape Indians} do vse to shoote with bowes, and ar­rowes % image:http://eebo.chadwyck.com/downloadtiff?vid=5891&page=7
 whiche are of a fadom in length, made of reedes, which grow in that countrie, and at the ende of them is artificially wyned a peece of wood very strong, vnto the whiche peece of woodde at the ende of it, they put a bone of a fish, in place of an arrow head: these kinde of bones be harder than Diamōdes and euery one of them be three or foure fingers long, and they are taken out of a fish that hath three of these bones, one vpon the backe, and one vnder euery wing: but that which grow­eth vpon the backe is the strongest and the greatest. In this Iland the people sayeth that there groweth golde, and in this Ilande and in {\itshape Paria} groweth reedes so great, that they make staues of thē, and carry of them into Spayne. Likewise there be Popingayes very great and gentle, and some of them haue their foreheades yellowe, and this sorte doe quickly learne to speake, and speake muche. There be likewise in the gulfe of {\itshape Paria} pearles, although not many, but very good and greate. {\itshape Paria} hath many Ilandes towardes the north parte of it, vn­till you come to the Iland of saint {\itshape Christopher,} and frō saint {\itshape Christopher,} to the Iland called {\itshape Espanola,} be other Ilands that lieth towards the Northeast. The names of these Ilands are as foloweth: The firste and nearest Ilande to {\itshape Paria} is cal­led the Iland of {\itshape Mayo,} this is but a small Ilande: there is a fruyte growing in it that the caske of it smelleth like to Ben­iamyn: 
	\normalmarginpar
      \marginnote{The Ilande of Mayo.} likewise there is Rosine in certayne trees, whiche they vse for Incense, and Aimasticke. This Iland stādeth in ten degrees and a halfe: the Iland of {\itshape Tabaco} in ten: {\itshape Santa Lu­cia} and the {\itshape Assention} in xj. degrees and a halfe: the {\itshape Baruada} in .xij. degrees: {\itshape Martinino} in .xij. degrees and a halfe: the {\itshape Dominica} in .xiij. and a halfe: {\itshape Gwadalupe} and the {\itshape Deceada} in xv. degrees: {\itshape Monserate} and the {\itshape Antigua} in .xvj. degrees: {\itshape Baruada} in .xvij. {\itshape Baruada} is compassed aboute with sholdes: 
	\normalmarginpar
      \marginnote{[...] {\gap •}andes.} the {\itshape Redonda} in .xv. degrees and a halfe: saint {\itshape Christopher} in xvij. degrees and a halfe: saynt {\itshape Bartolome} and saynt {\itshape Martin} in xviij. degrees and a halfe: {\itshape Sombrero} in xix. and a halfe: the {\itshape Anegada} in .xx. the {\itshape Virgines} in .xix. and .xx. {\itshape Sancta Crus} in % image:http://eebo.chadwyck.com/downloadtiff?vid=5891&page=7
 xviij. and a halfe: saynt {\itshape Iohn} in. xix and xx. and in .xx. and a halfe. From the {\itshape Trinitie} to saynt {\itshape Iohn} be two hundereth and fortie leagues. All the other Ilandes whiche I haue named, be in the middest of these twoo, and be all {\itshape Cannibals,} a people which eateth mans flesh, and they go to the sea in small botes 
	\normalmarginpar
      \marginnote{Canibal[...] {\gap •}.} called {\itshape Canoas} to make warre, one of them with an other, and as many people as they take one of an other, they carry to their owne Ilandes, and if they be men they eate them, and if they be women they serue them as slaues: and if any of the men that they take one of an other be leane, they put them to fatte, and when they be fatte they eate them, and they say that whiche is sweetest to be eaten in a man is the fingers, and the thinnest of the belly: these do vse to shoote with bowes and ar­rowes, and if they go to make warre, and do vnderstand that such as they go vnto, be stronger than they, then they leaue that place and goe to another. In all these Ilandes they say there is golde: in that of {\itshape Gwadalupe} hath bene golde found and gathered, but by reason they be not in subiection, there is no certaintie of it. The Iland of saynt {\itshape Iohn} is a good Iland, and 
	\normalmarginpar
      \marginnote{Saynt Iohn.} it hath two good portes: one of them standeth towardes the north, whiche is called porte {\itshape Rico,} and the towne that is in it is of Christians, and it is a good towne, although that it is not great: and the other is called saynt {\itshape German,} in this I­land is gathered much golde, and all is fine golde: this Iland is xxx. leagues long and lieth East \& Weast, and in breadth twentie. From this Ilande to the Ilande called {\itshape Espanola} are xvj. leagues, and the Ilande called the {\itshape Mona,} standeth well neare in the middest betweene bothe, whiche is a little Iland. 
	\normalmarginpar
      \marginnote{The Iland called Es­panalo.} At the beginning of the lande called {\itshape Espanola} is the Cape of {\itshape Higuei,} whiche standeth in .xx. degrees: from this cape of {\itshape Higuei,} vnto the cape of {\itshape Tiburon,} which is the cape and ende of the Iland, are one hundreth and sixtie leagues: they lie east and weast: from the cape of {\itshape Higuei} vnto the {\itshape Saona,} whiche is neare to the entring in at the porte of {\itshape Higuei,} are .xv. lea­gues. 
	\normalmarginpar
      \marginnote{Higuei.} The {\itshape Saona} lieth southweast in. x[...] {\gap •}x. degrees and a halfe. % image:http://eebo.chadwyck.com/downloadtiff?vid=5891&page=8
 Saint {\itshape Catherine} standeth in as many. From the entery of the porte of {\itshape Higuei} vnto the entry of the porte of saynt {\itshape Domingo,} the coaste lyeth weast. and are .xxxv. leagues: and from saynt {\itshape Domingo} to {\itshape Acuia} are twentie: {\itshape Acuia} lieth weast. From 
	\normalmarginpar
      \marginnote{Saynt Do­mingo.} {\itshape Acuia} to the {\itshape Beata} are twentie leagues. The {\itshape Beata} standeth from {\itshape Acuia} Southweast in xix. degrees. From the {\itshape Beata} to {\itshape Yaquimo} are fiue and twentie leagues. From {\itshape Yaquimo} to the {\itshape Cauana} are fourtie leagues. {\itshape Yaquimo} and the {\itshape Cauana} lieth in twentie degrees. From the {\itshape Cauana} lieth out a cape from the lande into the Sea towardes the South, cyght leagues, and from thence the coast turneth Weast Northweast vnto cape {\itshape Tiburon.} There is from one cape to the other .xxv. leagues. Before the {\itshape Cauana} standeth an Iland which is called [...] {\gap 〈◊〉} and betweene it and the {\itshape Cauana} is sholde, and the chanell li­eth neare to the lande, \& an other lieth at the end of the Iland. From the cape of {\itshape Tiburon} the coast turneth towardes the East vnto {\itshape Yaragua} three score and ten leagues, vntill it come neare to the Iland of {\itshape Guanabon.} In this three score and tenne leagues the Iland called {\itshape Espanola} hath not in breadth more than twentie or two and twentie leagues. From {\itshape Yaragua} the coast turneth towardes the Northweast and by North, vnto the cape of saynt {\itshape Nicholas.} There is from {\itshape Yaragua} to the cape of saynt {\itshape Nicholas} fiftie and fiue leagues, and the cape of saint {\itshape Nicholas} beyng doubled, the coast turneth towardes the east vnto porte Roiall: and from the cape of saint {\itshape Nicholas} to port Roiall be fifty leagues. This porte Roiall is the best porte of the Ilande sauyng that it is out of the way of all trade, and therefore they go not to it. From porte Roiall to the porte of {\itshape Plata,} are fiue and thirtie leagues: from the porte of {\itshape Plata} to the Cape {\itshape de Cierpe,} are fourtie leagues: the coaste lieth East Southeast. From the Cape of {\itshape Cierpe} vnto the cape of {\itshape Hi­guei} are eight and twentie leagues. And from the Cape of {\itshape Higuei} to the cape of {\itshape Tiburon} are one hundreth and eightie leagues, and that is the length of this Ilande. It hath in breadth from the {\itshape Beata} to porte Roiall ninetie leagues: the % image:http://eebo.chadwyck.com/downloadtiff?vid=5891&page=8
 South parte of it standeth in .xix. and .xx. degrees: the North parte in .xxiiij. This Ilād is inhabited with Christiās. There is gathered in it much gold: there cōmeth euery yeare frō this Iland to {\itshape Castile,} 400000. {\itshape Castellanos} \& more, euery {\itshape Castellano} is worth .vij. shillings of our money of Englād. It is a coun­trie of much fleshe, and also of much freshe fishe: the cattayle do multiply much, bycause there is no kinde of vermine that do hurt them. In al this Iland the Kine and Mares do bring forth yong ones, once euery yeare, and the yong Mares and 
	\normalmarginpar
      \marginnote{The man­ners of the Ilande cal­led Espa­nola.} yong Kine be with fole when they be but a yeare and a halfe olde. The grasse is alwayes greene and neuer waxeth drye: the trees be alwayes greene, with their leaues as they be here in the moneth of May and Iune. They eate bread of a roote which they call {\itshape Casaui.} There be other rootes like to Turneps, whiche be called {\itshape Aies,} and also {\itshape Batatas,} but the {\itshape Batatas} be bet­ter, and there be fieldes ful of them as be here of sowen fieldes. There is also a fruyte which is called {\itshape Pinas,} whiche be of the makyng of a Pine apple, but they be greater: the trees that beareth them be lyke to the Canes or stockes of Lillies, and {\itshape Flure de Luces.} Whē they be in season they turne yellow like to waxe: the smell of it is so much \& so sweete, y\textsuperscript{t} it smelleth in all y\textsuperscript{e} house where they be. It is of a maruelous goodly sauour although that the tast of it be somwhat egre. There is also an other fruit of trees which is called {\itshape Mameis,} which is as great, \& like to a Doues egge, of a tawny colour, being within three kernels, the meate of it is like to a Peach, somewhat red. The people of this Ilande were Idolaters \& naked, although that now they be turned Christians. The womē do weare a kind of apparel, which they call there {\itshape Naguas,} made in such sorte, that it couereth them from the waste to the knees: and they which be Virgines, go naked as they were borne: \& when any woman hath loste bir Virginitie, immediately she doth couer hir self, and if she haue no {\itshape Naguas,} she putteth before hir a leafe tied on with threedes made of Cotton woll, wherwith she co­uereth hir Secretes, and that leafe they call there {\itshape Pampanillia.} % image:http://eebo.chadwyck.com/downloadtiff?vid=5891&page=9
 and when any of these women are with childe, or giue sucke to any childe, there is no man that will company with hir for any thing in the worlde: they say it is sinne to company with hir at that time. And when any woman is with childe, hir husband taketh an other vntill his wife be deliuered of hir childe, and as long as she giueth sucke: and when any man dieth that is of estimation, they did make a greate hole in the ground, where they vsed to put him in, and they put him in sitting: and put in also with him both meate and drynke, and they couer the hole with timber and with earthe, and lefte open a straight mouth, where a man mighte goe in: to these came their wiues, and such as loued them well, and went in to him, and put in meate, and after that they were entered in, the mouth was couered with timber and earth: and so they were lefte all within, and they sayde that they wente to beare him company to the other world, where they should haue bet­ter cheere and more pleasure. And now there is in this Iland many townes of Christians, the principall towne is {\itshape Sancto Domingo,} where is a good porte and ryuer: and likewise there is the best trade of all the Ilande: \& this is towardes the south parte: \& at the north parte is the porte of {\itshape Plata:} but the towne is little, and the porte is not very good: and by this meanes the trade is litle. In this Iland be many moūtaines of Salte, and rockes of Salte. From the cape of Saint {\itshape Nicholas} to the Iland of {\itshape Cuba} are. xiiij leagues: the Iland of {\itshape Cuba} lieth west, it hath in length two hundereth and twentie leagues: and in 
	\normalmarginpar
      \marginnote{{\itshape Cuba.}} breadth by the cape {\itshape de Crus} fiftie leagues, and by the other partes by some wayes thirtie leagues, and other some twen­ty. It is a good Iland: and in it is much victuall, and much wilde foule, Pigeons, Geese and Partriche, and many Po­pingayes. The countrey is full of Mountaynes, there is ga­thered in it much golde, although that it is not so fine as that of the Iland called {\itshape Espanola.} The people be idolaters, the one parte of it is inhabited with Christians. The ende of the land which standeth next to the cape of saynt {\itshape Nicholas} is in .xxiiij. % image:http://eebo.chadwyck.com/downloadtiff?vid=5891&page=9
 degrees and a halfe, and that which lieth towardes the weast, is in .xxvij. degrees. The poynt of the cape {\itshape de Crus} is .xxiij. de­grees. 
	\normalmarginpar
      \marginnote{Cape de Crus.} This Iland of {\itshape Cuba} hath at the north parte of it aboue two hundreth small Ilandes, and they be all inhabited with 
	\normalmarginpar
      \marginnote{Small Ilandes.} people, which people be not very blacke, \& be of a good growth both the men and also the women: but there is no flesh to eate in these Ilandes: their meate is fishe, rootes, and bread made of rootes, and the blades of grasse: and if any of the people be caried to other places, if they doe giue them fleshe to eate, im­mediatly they die, if the flesh which they eate be not very litle in quantitie. Towarde the south parte of {\itshape Cuba,} is the Ilande called {\itshape Iamayca,} and this of {\itshape Iamayca,} lieth weast from the cape of {\itshape Tiburon.} There is from the cape of {\itshape Tiburon,} which is in the Iland of {\itshape Spanola} vnto {\itshape Iamaica.} xx. leagues, and in the middest 
	\normalmarginpar
      \marginnote{Iamayca.} betweene both lieth the {\itshape Nauaca,} whiche is a little small rocky Iland, beyng plaine and vnprofitable. {\itshape Iamaica} lieth East, and Weast, it hath in length fiftie leagues, and in breadth .xxv. it is a fruytefull Ilande, and hath muche corne growing in it, and hath all kinde of victuall, such as they vse to eate, whiche suffiseth their necessitie, and groweth vpon the same Ilande: there is much cattel, and many trees of cotton wooll: they ga­ther much Cotton wool in it, wherof they make store of cloth, but there is no golde: there is also great aboundance of Fishe. And there be a certayne kinde of little beastes, that haue the snoute and tayle like to a Ratte, and the body like to a Coney, and these be called {\itshape Hutias,} whose fleshe is good to eate, and there are multitudes of them. From the ende of the Ilande of {\itshape Cuba} towardes the Northweast, hath appeared a great coun­trie, it is thought to be a firme lande. And seyng that we haue spoken of the Ilandes, lette vs returne to the coast of {\itshape Paria,} 
	\normalmarginpar
      \marginnote{Paria.} where we began to speake of the Ilandes: I say, after that you goe from {\itshape Paria,} the coast of the lande turneth toward the Weast: there is from the mouth of the gulfe of {\itshape Paria} vnto the cape that standeth vppon the gulfe of {\itshape Aliosar} seuentie fiue leagues: the gulfe of {\itshape Aliosar} lieth Weast and by North, in % image:http://eebo.chadwyck.com/downloadtiff?vid=5891&page=10
 nine degrees and a halfe, and there remayneth in the middest the gulfe of all Sayntes: neare to the gulfe of all Sayntes towardes the weast there entereth a cape of a lande into the Sea three leagues: and neare to this Cape towardes the North eight leagues into the sea, there be Ilandes and rockes called the {\itshape Friers,} which be a company of small Ilandes lyke to rockes. And towardes the Weast parte from the {\itshape Friers} is the Ilande called {\itshape Margarita,} whiche is an Ilande that hath xx. leagues from the East to the Weast: and from the North 
	\normalmarginpar
      \marginnote{Margarita pearles.} to the South .xij. Round aboute al this Iland are fished fine pearles, in great quantitie. And in the gulfe of {\itshape Aliofar,} they fishe pearles also, but in this Ilande be more quantitie, and greater, and towardes the South side are moste and grea­test: and the Cape of {\itshape Aliofar} beyng doubled, a gulfe of Sea doth returne betweene twoo landes towardes the East, more than fiue and twentie leagues of breadth: in this gulfe be fi­shed many pearles, and muche {\itshape Aliofar:} the shelles wherein they breede be lyke to Cockle shelles, but they be greater and brighter within them, lyke to the selfe same pearles, but with­out they be of the colour of cockles: they fishe greate quan­titie of them. From the ende of the gulfe {\itshape Aliofar} vnto the Cape whiche standeth vpon the porte {\itshape Flechado} called {\itshape Tucu­raca,} he soure score leagues: {\itshape Tucuraca} standeth in nine de­grees and a halfe, and there remayneth in the middest the cape of small Ilandes, and before it is the porte called {\itshape Solo,} and the porte called {\itshape Canafistola,} and after him is porte {\itshape Flechado,} and hath many Ilāds before him: al this coast is of much fishing, and in it are trees of {\itshape Canafistola,} which doth bring forth there Canes so great as a great Launce, and are very good. And as it hath aboundance of substaunce, it dothe putrifie sooner than that which we haue here. The portes {\itshape Flechado} and {\itshape Canafistola} standeth in eight degrees. This countrey hath much victuall: 
	\normalmarginpar
      \marginnote{Cana fisto­la.} there is in it many Popingayes, Pearles, \& {\itshape Aliofar.} The {\itshape In­dians} of this countrey do vse to shoote with bowes, \& their ar­rowes be a fadom long: there are growyng great recdes as % image:http://eebo.chadwyck.com/downloadtiff?vid=5891&page=10
 big as a good staffe. Frō {\itshape Tucuraca} to the cape of saynt {\itshape Roman} are xlv. leagues. The cape of Saint {\itshape Roman} standeth South­weast and by South, in .xj. degrees: the cape of saynt {\itshape Roman} entreth into the sea .xx. leagues, and the lande is not past three or foure leagues broade, and towarde the cast lieth the port of {\itshape Coriana:} and towardes the weast porte of the {\itshape Pico,} they be good portes. There are from them to the cape twentie leagues, they are in ten degrees. From the cape of saynt {\itshape Roman} vnto the cape of {\itshape Coquibacoa} are three small Ilandes Trianglewyse, betweene these twoo capes is one gulfe of sea in shape foure square, and at the cape of {\itshape Coquibacoa} entereth in frō this gulfe an other little gulfe within the lande foure leagues, and at the 
	\normalmarginpar
      \marginnote{Coquiba­coa.} ende of him neare to the lande there lieth a great rocke, which rocke is plaine vpon the toppe, and vpō it standeth a village of houses of {\itshape Indians} whiche is called {\itshape Veneciuela,} and it standeth 
	\normalmarginpar
      \marginnote{Veneciue­la.} in ten degrees. Betwene this gulfe of {\itshape Veneciuela} and the cape of {\itshape Coquibacoa,} the water turneth within the lande towardes the Weast, and in this turne standeth {\itshape Coquibacoa.} Here hath bene founde wayght, and touche for golde, in the Towne, whiche is great: and the {\itshape Indians} doe say that they bryng the golde from within the lande, xxv. leagues, and when they goe thither they carry with them the waight \& the touch, whereby they know what they bring. In all the Weast {\itshape India} hath not bene found waight, but in this place. The towne of {\itshape Coquiba­coa} is greate, and a good towne, and of good peaceable peo­ple. In {\itshape Veneciuela} be people of a tall growth: and there are goodlier women than are in any other parte of that coun­trey. There are good Pearles, and well growen, although not so many as in the {\itshape Margarita.} Here the {\itshape Indians} do vse Laū ­ces of .xxv. foote long, and things to shoote withall like to dartes. From the Cape of {\itshape Coquibacoa,} vnto the cape {\itshape Dela Vela} are .xl. leagues. The cape {\itshape Dela vela} lieth weast northwest in 
	\normalmarginpar
      \marginnote{Cape Dela vela.} xij. degrees \& a halfe and neare to this cape {\itshape Dela vela} is a good port, with a little Iland before it, which lieth weast: \& beyng past cape {\itshape Dela vela} the coast turneth south \& by weast, \& lieth % image:http://eebo.chadwyck.com/downloadtiff?vid=5891&page=11
 lowe, and likewise all the land from the cape {\itshape Dela Vela} is low lande. From the cape {\itshape Dela Vela} to {\itshape Tucuraca} are .xxx. leagues: {\itshape Tucuraca} standeth in .xj. degrees and a halfe, \& is a good porte. Frō {\itshape Tucuraca} to {\itshape Sancta Marta} are .xxv. leagues: {\itshape Sancta Mar­ta} lieth weast in. xj degrees and a halfe, and is a good porte: 
	\normalmarginpar
      \marginnote{Tucuraca.} it hath a small Ilande before it, \& is the best porte of all this coast. This countrey of {\itshape Sancta Marta} is a countrey that is watered by sluces, and the corne and things which they sowe 
	\normalmarginpar
      \marginnote{Sancta Marta.} and plant, they doe water, with their owne labour: it is an o­pen grounde, and hath hight mountaynes, and without any thing growing in them. The sande of the riuers is altogether stony sande, of the colour of golde, and it sheweth as though that it were al gold. There are in this Ilande many Hogges, and much Deare. There is found in the powre of the {\itshape Indians,} much gold and coper, and also much gilt Coper. The {\itshape Indians} do say that they gilte the Coper with an hearbe that grow­eth in that countrey, whiche beyng stamped, and the iuyce ta­ken out, and the Coper beyng washed therewith and put to the fire, it turneth to the colour of most fine gold, and it riseth more or lesse in colour, accordyng to the quātitie of the hearbe that is put to it. The people be fearce and warlike, they vse bowes, and their arrowes be little bigger than quarrels, which they annoynt with an hearbe that is so full of poyson, that it is a great chaunce (when they hit any man) if he escape, beyng wounded with that hearbe: one of the things wherewith they make that hearb, are certaine apples which be in that coūtrey, 
	\normalmarginpar
      \marginnote{Appels of poyson.} \& are of the greatnesse and colour of the wilde Apples of this coūtrie. The tree that doth carry them is like to a small Peare tree of this coūtrie. As sone as any mā hath eaten one of thē, immediatly it turneth to wormes within his body, whiche grow so great \& eate so muche of the bodie, that they kill him: this is knowen to be so by reason. I caused it to be experimen­ted, in one that was giuen to a Dogge to eate, whiche within foure houres was turned all into wormes in the body, and so we found it when we caused the Dogge to be opened, for to % image:http://eebo.chadwyck.com/downloadtiff?vid=5891&page=11
 see the effect of the Apple. And if any man do put himselfe vn­der one of these trees in the shadowe, immediatly his head be­ginneth to ake: \& if he do continue there long, his face begin­neth to swell, \& to trouble his sight: and if by chaunce he sleepe vnder the tree, immediatly he loseth his sight: all this I haue seene by experience. Before you come to {\itshape Sancta Marta} there is {\itshape Yaharo,} which is in the side of the snowie mountaynes. {\itshape Ya­haro} as a good port and standeth in a good countrey: and there 
	\normalmarginpar
      \marginnote{Yaharo.} be Orchardes of trees of diuers sortes of fruytes to be eaten, \& among them there is one like to Oringes, \& when the fruite is in season to be eaten it turneth yellow, and the meate that is within it is like to butter: It hath a good sauour, and hath the tast so good \& so soft, that it is maruelous. The moūtaines that lie full of Snow, beginne from {\itshape Sancta Marta:} and neare to {\itshape Yaharo} is the highest parte of them, and they shew vpon the toppes as white as if they were Snowe, and from thence they reache to {\itshape Veneciuela,} and so into the countrey within, it is not 
	\normalmarginpar
      \marginnote{Moūtayns vvith Snovve.} knowen how farre, for that the coūtrie is not yet in subiectiō, nor the {\itshape Indians} do giue any more reason of them, but that they reache very farre into the countrie. This Mountayne is vpō the toppe playne, and there be many townes of {\itshape Indians} vpō it, and many lakes of standing water. In {\itshape Sancta Marta} is ga­thered much Cotton wooll, and the {\itshape Indians} do make store of clothe of it, whiche is a thing to be seene, and they make it of many colours: they do make of Popingayes fethers, Pecocks and of other birdes which be in that countrey, certaine things like to Diademes, very great, that the gētlewomen do put on their heades, which things hang downe vpon their shoulders in one peece vnto their girdle: like to the ends of a bishops Mi­tre: \& this is so well wrought, that it is maruell to see the di­uersitie of the colours, and the worke, and the arte thereof: and as the colours be naturall and of themselues, they shewe so wel, that no artificiall worke of such as is wrought here, is so good nor so acceptable vnto the sight. From {\itshape Sancta Marta} the coast turneth south .xx. leagues: and in the turnyng cape % image:http://eebo.chadwyck.com/downloadtiff?vid=5891&page=12
 of {\itshape Sancta Marta} standeth {\itshape Garia,} where be a naughty kinde of people: and righte before this wilde countrey entreth in­to 
	\normalmarginpar
      \marginnote{Garia.} a Riuer very greate, whiche goeth from the Mountaynes that lie full of Snowe: and he is so greate, that thys wa­ter entreth a greate way into the Sea without any min­glyng of it selfe with the salte water: and from thence the coast lieth Weast, vnto the porte of {\itshape Zamba. Zamba} is a good porte, and standeth in eleuen degrees and a halfe. From {\itshape Sancta Marta} to {\itshape Zamba} be fiue and twentie leagues: the lande of this coaste is playne, and lowe without Moun­taynes, and it is all very fayre medowe grounde, and a countrie well inhabited: the menne haue their heare cutte, the women goe couered from the waste downeward. They are good people and burte no bodie, but suche as doe hurte them firste. {\itshape Zamba} hath towardes the Weast parte of it foure I­landes of sandes, whiche lyeth neare the lande, and rounde aboute them be sholdes: they reache tenne leagues into the 
	\normalmarginpar
      \marginnote{Ilandes of Sandes.} Sea, but betweene them and the lande where the Cape {\itshape De Loyo delo gato} standeth, there may passe shippes from {\itshape Zam­ba} vnto the Cape of {\itshape Canoa} whiche is twoo leagues. From {\itshape Carta gena} be twenty leagues, and they be all sholdes of the Ilandes of sandes. Before the Cape of {\itshape Canoa} there is a rocke whiche ryseth a little aboue the water, whiche rocke they call {\itshape Canoa,} but by reason it is seene, it is not daungerous. And a little before aboute twoo leagues from thence, are the portes of {\itshape Carta gena} The portes of {\itshape Carta gena} haue an Iland 
	\normalmarginpar
      \marginnote{The portes of Carta gena.} in the myddest, whiche standeth not out of the compasse of the other lande, and by the one and the other side of thys Ilande, there is a porte, but that of the Easte parte hath the better entrie, the Ilande is called {\itshape Gnodego:} hee hath twoo leagues in length, and halfe a league in breadth, and is well inhabited of {\itshape Indians} beyng fisher menne. The peo­ple of this Countrey be tall, but bothe menne and women goe all naked as they were borne: they are warrelyke peo­ple, and vse bowes and arrowes: they shoote all their ar­towes % image:http://eebo.chadwyck.com/downloadtiff?vid=5891&page=12
 wyth a naughty hearbe, and the women doe lyke­wyse fighte as well as the menne. I had prysoner a gyrle of eyghten or twentie yeares of age, who dyd affirme that she had kylled eyght Christian menne before shee was taken prysoner. In this place groweth the hearbe {\itshape Iperboton,} 
	\normalmarginpar
      \marginnote{Iperboton▪} wherewith they heale the woundes of the hearbe whiche is poyson, and with thys Hearbe they say that {\itshape Alexander} hea­led {\itshape Ptholome.} In this Countrey and towardes the Weast partes the {\itshape Indians} doe eate breade of the grayne of {\itshape Maiz,} grounde: and they make of it good bread, whiche is of muche substaunce: and of the selfe same meale of {\itshape Maiz} beyng sod­den in kettels and great tinages in muche water, they make wyne to drynke: and it is wyne o[...] {\gap •} much substaunce, good, and of a good sauour. The {\itshape Indians} doe vse to drynke a greate cuppe full of it when they ryse in the mornyng with­out eatyng any thyng else with it, and therewith they goe to theyr dayly laboures, and be there labouryng the grea­test parte of the day, without eatyng any other thyng. The Christians that are in that Countrey doe vse the lyke, and they say that it is there in that Countrie: and a manne endureth to laboure with it all one daye without eatyng any other thyng, if he drynke twoo tymes thereof.\par
The hearbe {\itshape Iperboton} wherewith they heale the hearbe that is poyson, they say that the iuyce of hys toote is as good for the sighte as for to heale the hearbe of poyson, and that there is of it in {\itshape Carmania,} and in the Mountayne {\itshape Ata­lantes,} whiche are in the coaste of {\itshape Getulia.} In this countrey of {\itshape Carta Gena} is in the power of the {\itshape Indians} muche Copper▪ and there is lykewise golde, but not muche: and they say that twentie leagues from that Countrie towardes the South­weast is muche golde, and whosoeuer will may goe thither for it.\par
From {\itshape Carta gena} to the Ilandes of {\itshape Caramari} whiche lieth 
	\normalmarginpar
      \marginnote{Cara[...] {\gap ••••}.} Weast, are eyght leagues, these Ilandes be all lowe, and no shippe can passe betwene them. From the Ilands of {\itshape Caramari,} % image:http://eebo.chadwyck.com/downloadtiff?vid=5891&page=13
 vnto the Ilandes of {\itshape Baru} be ten leagues: betweene these of {\itshape Baru} and the land may passe shippes if they be not very great, and beyng paste these Ilandes of {\itshape Baru,} more towardes the Weast is the porte of {\itshape Cenu,} which is a great Baye, and hath his entry by the East part, and it is a good sure porte. There is 
	\normalmarginpar
      \marginnote{Cenu.} from {\itshape Carta gena,} to {\itshape Cenu.} xxv. leagues: {\itshape Carta gena} lieth East in ten degrees and a halfe, and {\itshape Cenu} towardes the Weast in ix. degrees: in that of {\itshape Cenu} they make muche Salte: the peo­ple be strong and warlike: they vse bowes, and their arrowes be set with poyson. Ill men and womē goe naked. When a­ny man of great auctoritie dye, or any chylde of hys, they take their guttes out of their bodies, and washe them with certaine thyngs, and annoynt them, and vpon them they put Cotton wooll, died with diuers colours, whiche they put againe into the bodie, and beyng coloured therwith, they put them into a bedde made after the fashiō of the beddes of that countrey, and they hang him vp in the house, neare to the place where they make their fire: and so they keepe him It happened to me, that I tooke a towne called {\itshape Catarapa,} where we founde more than xx. that were hanged after this sort in the houses. In this coū ­trey of {\itshape Cenu} is much golde in the power of the {\itshape Indians,} \& very fine, and it is myngled and hath his foundation of Siluer, and there is no parte of it Copper, whiche the {\itshape Indians} doe say they bryng from certayne Mountaynes, from whence the ryuer of {\itshape Cenu} doth come forth, from certayne places that they call {\itshape Mocri,} and an other {\itshape Cubra,} and an other {\itshape Cud[...] {\gap •}:} and the earth that they haue in those places is somewhat redde, and 
	\normalmarginpar
      \marginnote{An earth of golde.} they gather it in riuers \& valleys: and when it rayneth they caste nettes ouerthwart the riuers \& valleys, and as the wa­ter increaseth, it bringeth graynes of gold as great as an egge \& they remayne in y\textsuperscript{e} nets: \& in this sorte they gather the grea­test graynes: \& y\textsuperscript{t} which they gather, they bring to the towne which is called {\itshape Cenu,} being .x. leagnes from the Sea vpō the riuer, \& there they worke it, and doe what they will with it. I toke an {\itshape Indian} Gentleman prisoner, y\textsuperscript{t} sayd to me, that he had % image:http://eebo.chadwyck.com/downloadtiff?vid=5891&page=13
 gone to this place three times, and had seene it gathered after that sort, and also that he himselfe had gathered it. This coun­trey of {\itshape Cenu} hath great store of vittayle growen in the Coun­trey. Their bread and Wyne is made of the meale of {\itshape Maiz,} as it is in {\itshape Carta Gena.} Likewise there be rootes, whereof they make bread, as in the Iland of {\itshape Cuba,} and {\itshape Iamaica,} and the {\itshape Es­panola,} but it is of another qualitie: for that of the Ilandes is naught, and if any do eate of it, he dyeth, as though he had ea­ten {\itshape Arnike:} and also if any beast eate of it, or drink of the wa­ter that commeth out of it, he dyeth. And to make bread of it, they grate it, and after they presse it, and when it remayneth as drye as powder, they make bread of it: and that which gro­weth in this Citie of {\itshape Cenu,} and in all y\textsuperscript{e} Countrey heereabout, they eate them rawe, and rosted, for they are very good to be so eaten, and of a good sauoure.
\subsection[{A Protestation.}]{A Protestation.}\par
I Did require two {\itshape Indian} Gentlemē of {\itshape Cenu,} in the behalfe of the King of {\itshape Castile,} that they shoulde be subiect to the Kyng of {\itshape Castile,} and that he woulde giue them to vnderstande, that there was one God alone, whyche was three persons in one, and did require them, that they woulde leaue vnto him that Countrey, seeyng that it did apperteyne to hym: and if they would liue in it as they were, that they should giue to him the obedience, as vnto their Lord, and also shoulde giue him in to­ken of obedience, some thing euery yeare, euen so much as they themselues would name, and if they would this doe, the Kyng would giue them rewards and giftes, and helpe against theyr enimies, and would put among them learned men \& Priestes, that should shew them that which doth apperteyne to the faith of Christe: and if any of them woulde not turne Christians, they should not be compelled by force, agaynst their willes, but they might remayne as they were before, and they answe­red me to that I sayde that there was but one God, and this God gouerned the Heauen and the Earth, and was Lorde of % image:http://eebo.chadwyck.com/downloadtiff?vid=5891&page=14
 all. It lyked them very well, who sayde, y\textsuperscript{t} they thought it to be true: and they sayd, that they were Lords of their owne Coū ­trey, and had no neede of any other Lorde. Then I required thē agayne y\textsuperscript{t} they should do it, \& not doing it, I would make warre agaynst them, \& would take their Towne, \& would kill as many as I tooke, or would keepe thē prisoners, and sell thē for slaues. They aunswered me, that first they would put my head vpō a staffe, \& they laboured to do it, but they coulde not, for we tooke the towne by force, although they shotte at vs an infinite nūber of arrowes, \& al poysoned w\textsuperscript{t} hearbes, wherwith they wounded two of our men, and both dyed, although their wounds were but small. After I tooke prisoner an {\itshape Indian} Gentleman, of those w[...] {\gap •}ich I spake of before, that had declared to me of the mines of {\itshape Vocri,} whome I founde to be a man of much truth, \& kept his promise in al things. And after this sort are all the warres of those Countreys. Frō this riuer of {\itshape Cenu,} vnto the Gulfe of {\itshape Vraba,} are. xxv leagues. The gulfe of {\itshape Vraba} lieth towards the Weast in .viij. degrees. Al this Countrey is full of Mountaynes, \& a naughtye kinde of people: they are all {\itshape Canibals,} \& eate mans flesh. They vse to shoote with bowes and arrowes which are poysoned. Fiue leagues from the Riuer of 
	\normalmarginpar
      \marginnote{Canibals.} {\itshape Cenu} towards the West, is an Iland called the strōng Iland, welnecre a league frō the lande. In this Iland is muche salte made, and neerer the Gulfe is another, called the {\itshape Tortuga.} The Gulfe of {\itshape Vraba} hath. xiiij leagues of length within the land, \& 
	\normalmarginpar
      \marginnote{The Gulfe of Vraba} of breadth in the mouth an entrie .vj. or .vij. leagues, farther in fiue, and neere the cape foure. At the entrie toward the East he hathe certaine sholdes, which entreth in more thā two leagues into the Sea athwart of the mouth \& entrie, and they goe wel­neere ouer the one halfe of the entrie towards the Weast part of the Gulfe. And there is fiue leagues within the Gulfe, the {\itshape Darien,} whych is inhabited with Christiās, and there they ga­ther fyne golde in a Riuer that descendeth from certaine high Mountaynes. In these Mountaynes be many Tigres \& Ly­ons, and diuers other beastes, and Cattes with lōg tayles, and % image:http://eebo.chadwyck.com/downloadtiff?vid=5891&page=14
 be like to Apes, but that they haue great tayles. There are also Swyne, \& many great beastes as great as kyne, fatte, whiche be of a grey couloure, and haue their feete like to kyne, the head like to a Moyle, with long eares, their flesh being good to eate. There be also many other beastes. I toke that Towne, which was the first that was taken in that Countrey, and I saw all these beasts, and it was told me, that they had sene Ownees. I saw none, but I saw a riuer y\textsuperscript{t} passeth by the towne of {\itshape Darien,} wherein were many Lyzards that were great, \& so grose in the body as a Calfe: and if they see any man, dogge, or pigge neere the water, they come out of it, and fall vpon him, and if they catche him, they carrie him into the water, \& eate him. I hap­pened to kill y\textsuperscript{e} first that was killed, \& I saw cast at hym more than tenne Launces, and as they strake vpō him, they reboun­ded, as though they had stricken vpō a Rocke. And after that, a seruant of mine went athwart of hym, and thrust him at one blowe into the middest of his body, and then we killed hym, and being dead and taken on lande, we found that he had vpon his backe from his necke to the tayle a shell that couered hym all, whiche was so strong, that no Launce coulde passe it: and vnder that from the middle of the body downewarde neere to the guttes, he was as other Lyzards are, and by that parte of his body he was thrust in with the Launce. His mouth was three spannes long from the snoute to the lower end of y\textsuperscript{e} teeth: he had in a side two rewes of teeth, the most fierce that I haue seene, or had bin seene by any of them that were with me. He was fleyne \& his flesh was eatē, it was white \& good, \& smelled like to muske, and good in eating. I saw also y\textsuperscript{e} flesh of Tigres \& of Lions. I saw certayne mē kil Lyons alone by thēselues. 
	\normalmarginpar
      \marginnote{Lyons.} The Tygres are greater of bodye than the Lyons are, and 
	\normalmarginpar
      \marginnote{Tygres.} their feete are verye strong, and of greate force, but they are heauie, for they runne slowly \& are of little courage. It hapned y\textsuperscript{t} a Tygre wēt after a mā a league, vntil he came to a towne, \& the mā neuer wēt but his accustomed pace, \& the Tygre af­ter him .iij. or .iiij. speares lēgth bebind, \& in a league y\textsuperscript{t} they wēt % image:http://eebo.chadwyck.com/downloadtiff?vid=5891&page=15
 togither, he durst not to fasten with the man. The Lizardes in 
	\normalmarginpar
      \marginnote{Lizards.} the moneth of Ianuary and February do breede in this order fellowyng When the Sunne is hotest in the day, they come for the of the water into the sandes, and with their foure feete they make a hole, and there lay theyr egges, and after they bee layd, they couer them with the sand, and with the heate of the Sunne. The Lizardes be ingendred in those egges, and after they make a hole in them, and they come forth of the egges in­to the sandes, and so they goe into the water. The egges be as greate as a Goose egge, and greater. They haue no shelles, but certayne skynnes. They be good to eate, and of a good sauoure, \& euery Lizard doth lay at one tyme lx. or lxx. egges. Lyke­wise there be others called {\itshape Yaguanas,} whiche be great, and lyke 
	\normalmarginpar
      \marginnote{Yaguanas.} to Lizards, and these haue round heads, and from the forepart of the head to the tayle, he caryeth brustels of heare, standyng vp very fierce. They be of a russet coloure, and somewhat in coloures: these goe into the Mountaines. The sight of them is fearefull, but they hurt no body, by reason they take them a­liue, and kill them with staues. These are good meate, \& theyr fleshe is muche esteemed in that countrey. Their egges be of a good tast. In these Countreys is muche wilde foule of dyuers sorts, which are good flesh. There are abūdance of greene Po­pingeys, and some greate ones of many coloures, red, blewe, blacke, and greene, whiche are faire to beholde. Their fleshe is good and sweete: and others that are as little as grey Spar­rowes, which be greene and faire. In this countrey are greate fishings of good fish: and also there be Date trees that do carry frute as greate as a Doues egge, some yellow, and other of the Rose coloure, but they haue greate stones: their tast is somewhat sharp. In this Countrey are little beastes like to Pigges of a month olde, these haue their feete and heads lyke to a little Horse, with little eares, and they be all couered with a shell from the eares to the tayle, and be like to a Horse coue­red. They are faire to behold. They feede like to Horse. In this countrey are Conies and Partriches, and many goodly birds. % image:http://eebo.chadwyck.com/downloadtiff?vid=5891&page=15
 And the bread and Wine of this Countrey is made of {\itshape Maiz,} (as it is saide:) The people are tall of growth, and Idolaters. There are some of them that beleeue, that there is no other thing, but to be borne▪ and to die. There be amōg them Lords, whome they honoure much. And he that is a great Lord, they call {\itshape Tiba,} and others that be not so great, {\itshape Quin[...] {\gap •}s.} Before thys Riuer of {\itshape Darien,} entreth in another Riuer very greate in thys gulfe of {\itshape Vraba,} and he entreth in by sixe or seauen monthes, 
	\normalmarginpar
      \marginnote{A greate Riuer.} although they be but sholde, and no shippe can enter in at thē, if they be greater than small boates: but within the mouth he is greate, and fifteene and twenty fadome deepe, and a myle broade, \& hath abundance of water, by reason that xl. leagues within the land there ioyneth with him greate riuers, whiche commeth from the East parte of the Mountaynes, where the Riuer of {\itshape Cenu} springeth: and the first riuer that ioyneth wyth him, is that of {\itshape Dabayne.} In the springs of this Riuer, and of another whiche is before this, they saye that there are greate mines, but the truth is not knowen, but the {\itshape Indians} doe say it. And there hathe bin takē in the power of {\itshape Indiās,} peeces of fine golde, that wayed seauen and eyght hundred waight. In the little riuers of this great riuer, are many places drowned vp, and in them are many {\itshape Indians,} and haue their houses and ha­bitations vpon trees, for that vnder them is al water, and they liue by fishing. This gulfe of {\itshape Vraba} hath on the other parte of the Mountaine of {\itshape Darien} towards the South, another gulfe, called the gulfe of {\itshape Saint Michaell,} and there is frō the one to y\textsuperscript{e} other xxv. leagues \& more. There is this differēce betwene thē: 
	\normalmarginpar
      \marginnote{The Gulfe of S. Mi­chaell.} in the gulfe of {\itshape Darien} and {\itshape Vraba,} the Sea dothe not flowe one palme, and in that of {\itshape Saint Michaell,} it floweth as much as in {\itshape Britaine.} And of this coast of the gulfe of {\itshape Saint Michaell} I wil speake hereafter. And now I do returne to the gulfe {\itshape Vraba,} and do saye, that from the entrye in of the gulfe of {\itshape Vraba,} vnto the port {\itshape De Careta,} are fifteene leagues. {\itshape Careta} lieth Northweast, 
	\normalmarginpar
      \marginnote{Careta. Nombre de Dios.} in nine decrees and halfe. From the Port {\itshape Perdido,} to {\itshape Nombre de Dios,} the coast lyeth West North West. {\itshape Nombre de Dios} % image:http://eebo.chadwyck.com/downloadtiff?vid=5891&page=16
 standeth in tenne degrees and halfe, and there are in the mid­dest {\itshape Conegie, Pocurosa,} and the baye of {\itshape Saint Blase.} There are from the port {\itshape Perdido} to {\itshape Conogie} seauen leagues. From {\itshape Cono­gie} to {\itshape Pocurosa} tenne. From {\itshape Pocurosa} to the baye of {\itshape Saint Blase,} fyue. From the baye to {\itshape Nombre de Dios} sixe. In all this coun­trey 
	\normalmarginpar
      \marginnote{Pocurosa.} they call men {\itshape Omes,} and the women {\itshape Iras.} The men goe naked, and they vse to bring tyed at their middle with a small cord certayne Snayles shelles of the Sea, and into these shels they put their members, and some doe carrie a thing like to a fonnell of golde, wherein they put their members. The Wo­men goe all couered from the wast downewarde with wast coates of cotten woll, and weare rings put through their eares, and many other things, and cheynes of gold. There hathe bin found amongst the {\itshape Indians} much golde, although that muche of it is course, and in valewe of tenne and twelue kuyllats and lesse, and they call this {\itshape Giamin.} In y\textsuperscript{e} South part of this coun­trey is found golde in the Riuers, and as it hath not bin much sought till now, it hath not bin knowen. I haue seene a graine gathered in a Riuer, whiche wayed seauen Ducates. From {\itshape Nombre de Dios} to {\itshape Veragua,} are fiue and thirtie leagues. {\itshape Vera­gua} is 
	\normalmarginpar
      \marginnote{Veragua.} towardes the Weast in tenne degrees. And neere to {\itshape Nombre de Dios} is Port {\itshape Bello,} which is a good Port. It hathe at the entrie of it a little Ilande, and in the middest another. The Countrey of this coast is full of sharpe Mountaynes, and barren. From {\itshape Veragua} the coast turneth North to the cape of {\itshape Gracias Adios,} threescore leagues. The cape of {\itshape Gracias} 
	\normalmarginpar
      \marginnote{The cape of Gracias Adios.} {\itshape Adios} standeth in foureteene degrees. And neere to {\itshape Veragua} is the {\itshape Furmia,} and after {\itshape Corobora,} and after that certayne I­landes, compassed aboute with sholdes: and from the cape of {\itshape Gracias Adios} the coast turneth Weast▪ to the cape of {\itshape Caxi­nes,} where is an Ilande, and is thirtie leagues distante. And from the cape of {\itshape Caxines} the coast turneth towards the North threescore and fifteene leagues, vnto the cape {\itshape de Lagar,} and all these threescore and fifteene leagues be sholdes, and entreth in­to 
	\normalmarginpar
      \marginnote{Sholde[...] {\gap •}.} the Sea towardes the East threescore leagues: and from % image:http://eebo.chadwyck.com/downloadtiff?vid=5891&page=16
 this cape to the cape of the Iland of {\itshape Cuba,} are one hundred and twenty leagues. The cape of {\itshape Cuba} lieth North. From the cape of {\itshape Lagar,} vnto the cape of {\itshape Fondura,} are fiue and thirtie leagues. The coast lieth West. The cape of {\itshape Lagar,} and that of 
	\normalmarginpar
      \marginnote{The cape Lagar.} {\itshape Fondura,} standeth in .xvj. degrees, and being past the cape of {\itshape Fondura,} is a Gulfe that entreth .xv. leagues within the lande, and in the entrie it hath .xviij. leagues of length, and farther in xxv. From this cape, to the bay of {\itshape S. Thome,} are .lx. leagues, and the coast lieth Weast. And neere to the bay entreth one end of 
	\normalmarginpar
      \marginnote{The bay of S. Thome.} the lād into the sea .xx. leagues towards the Northweast, and at the end of the land is the entrie of the baye of {\itshape S. Thome,} and y\textsuperscript{e} bay doth returne vpon the cape Southeast .xlv. leagues, and carieth in length and breadth .xv. leagues. And at the entrie in of this gulfe towards the Weast, appeareth another little roūd gulfe full of small Ilands. From the mouth of this gulfe, to the Port of the {\itshape Figueras,} are .xxx. leagues. The coast lieth North­weast, and it is all sholdes. The entrie of the Gulfe of {\itshape S. Thome} standeth in .xix. degrees and halfe. It entreth in at the East part, for by the Weast part it is all sholdes. The cape of the {\itshape Fi­gueras} standeth in .xxj. degrees, and from this place the disco­uerers 
	\normalmarginpar
      \marginnote{The cape of the Fi­gueras.} returned, and past no farther: \& they found in this Coū ­trey trees of oke, with acornes like to oures, and many birdes different from those of this Countrey: and hennes as greate as Geese: and they found tokens \& shewes of much golde, for they found great peeces of golde in the power of {\itshape Indians.} And now I do returne to the Gulfe of {\itshape S Michael,} \& to the coast of y\textsuperscript{e} other 
	\normalmarginpar
      \marginnote{The Gulfe▪ of Saint Michaell.} side towards the South. The Gulfe of {\itshape S. Michael} lieth North­east \& Southweast, with y\textsuperscript{t} of {\itshape Vraba,} there are .xxv. leagues of land betweene the one and the other. That of {\itshape S. Michael} is to­wards y\textsuperscript{e} Southweast, \& it hath at the entrie in .x. leagues, \& of length .xxv. And towards the Southweast lieth the Ilande of {\itshape Perles.} There is from the Gulfe to the Ilande tenne leagues. 
	\normalmarginpar
      \marginnote{An Ilande of Pearles.} This Ilād is plentiful of vittaile. There is in it many birds, \& so many conyes, that they enter into the houses to breede. And round about it on euery side, is much fishing of great Pearles, % image:http://eebo.chadwyck.com/downloadtiff?vid=5891&page=17
 and very fyne. I sawe one of the fashion of a little peare, the best and greatest that I haue seene. The entrie of the Gulfe of {\itshape Saint Michaell,} standeth in sixe degrees. From the entrie of the Gulfe of {\itshape Saint Michaell,} the coast lieth West, seauen leagues vnto the Cacike {\itshape Tamao.} The coast lyeth .xxviij. leagues to­ward the Northweast, and by Weast, to the Riuer of {\itshape Tuba­nama.} 
	\normalmarginpar
      \marginnote{Tamao.} This Riuer of {\itshape Tubanama} hath at the entrie an Ilande 
	\normalmarginpar
      \marginnote{Tubunama} in triangle, whiche causeth him to haue two mouthes: it is a good Riuer, and a good Countrey, well inhabited, and of much vittayle, and great fishing: there is in it much golde. In this coast the Sea doth ebbe and flowe much. From this Ri­uer to {\itshape Panama} are twenty leagues, and the coast lieth Weast, and there is in the middest the Riuer called {\itshape Chapanere,} and the 
	\normalmarginpar
      \marginnote{Panama.} Riuer {\itshape Pacora,} whiche are in a good countrey, and where as is 
	\normalmarginpar
      \marginnote{Pacora.} gold, and the people are good. {\itshape Panama} hath an Iland towards the Sea neere y\textsuperscript{e} land. Frō {\itshape Panama} to the cape of {\itshape Chiru} are .xx. leagues, and the coast lyeth Weast and by South, \& there is 
	\normalmarginpar
      \marginnote{Chiru.} in y\textsuperscript{e} middest {\itshape Perequete} and {\itshape Tabora.} From {\itshape Chiru} towards the West appeareth a gulfe oblique, in the middest whereof dwel­leth the Cacike {\itshape Paris,} an {\itshape Indian} Gentleman of greate power, and the richest in that coast: and there remayneth in the myd­dest 
	\normalmarginpar
      \marginnote{Paris.} {\itshape Nathan} and {\itshape Estora,} which are good townes. All this coast frō the Gulfe of {\itshape Saint Michaell} to {\itshape Chiru} and {\itshape Paris,} be from sixe to seauen degrees. The countrey is playne and good, and of much vittayle of all sortes, and in all that countrey is muche golde. From {\itshape Paris} lyeth a poynt of the land into the Sea twē ­tie leagues, and being past the poynt, the coast turneth to the Northeast, vntill it come hard to the cape of {\itshape Gracias a Dios,} and all the Countrey is frutefull, and a rich Countrey of gold, whereas are many good townes. In this Countrey they doe compasse about the townes with timber, for feare of Tygres, and Lions, that they come not by night into the houses. From the Gulfe of {\itshape Vraba} \& {\itshape Saint Michaell,} to the end of the cape of {\itshape Gracias Adies,} are two hundred leagues: and all this Coun­trey hathe not in breadth more than thirtie, and where it is % image:http://eebo.chadwyck.com/downloadtiff?vid=5891&page=17
 most fourtie leagues, and all are good people and riche, and a fruyteful coūtrey. There they vse no bowes, nor hearbe of poy­son, but Launces and Dartes. And there are in this Sea to­wardes the Southe many Ilandes, where is as muche golde and Pearles, as the {\itshape Indians} doe say. And they say that there is a countrey where the people that doe inhabite it, haue bookes, and do wryte and reade as we doe.\par
Now seyng that we haue spoken of that part that is from the Ilande of {\itshape Fierro,} towardes the Weast, and Northweast, lette vs speake of one peece of lande whiche is in this seconde parte towardes the North, whiche lande is called the lande of {\itshape Labrador.} This lande of {\itshape Labrador} standeth in .lvij. degrees: it standeth Northweast from {\itshape Calizia,} and East and Weast with Scotlande. This lande of {\itshape Labrador} hath towardes the Weast parte of it the countrey of {\itshape Bacallaos,} whiche is a coun­trie of muche fishyng, and a great countrey: and the {\itshape Bocallaos} standeth Weast from {\itshape Galizia,} and parte of them Weast and by Northe, and this countrey hath many portes, and good: Muche of the countrey is inhabited, and there are many Ilandes before it, all inhabited. They say there is in it greate quantitie of Furres, and very fine. The lande of {\itshape Labrador} is towards the North from the {\itshape Acores.} There is from the {\itshape A­cores} to the lande of {\itshape Labrador} three hundreth leagues, and from {\itshape Galizia} to the lande {\itshape Labrador} three hundreth and fiftie. And there is from {\itshape Galizia} to the lande of {\itshape Cauallaos} fiue hundreth and thirtie leagues. The lande of {\itshape Cauallaos} standeth in fourtie nine, and in fiftie degrees.

\begin{raggedleft}FINIS.\end{raggedleft}

\end{document}
